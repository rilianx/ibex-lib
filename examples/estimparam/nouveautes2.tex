\documentclass{article}
\usepackage[boxed,lined]{algorithm2e}
\usepackage{algorithm2e-cgf}

\title {Estimation de param\`etres ;  nouveaut\'es apr\`es ICTAI 15}
\author{Bertrand Neveu}
\begin{document}
\maketitle

Voici les principaux changements par rapport \`a la version de l'article d'ICTAI

\section{Condition d'arr\^et}


{\bf Arr\^et d'une branche dans version ICTAI}
   Condition d'arr\^et pour la branche courante (pas de bissection) dans ICTAI : pr\'ecision atteinte ou possibles = valides.
 La gestion des solutions partageant des observations valides a lieu dans le post-traitement et on ne garde
que les solutions maximales, ne partageant pas plus de la moiti\'e de leurs observations avec une autre solution maximale.(cf partie 8)

{\bf Nouvelle condition d'arr\^et}
   Nouvelle condition :  on s'arr\^ete quand il ne reste pas plus de Q/2 points possibles non valid\'es : 
   toute autre solution dans la bo\^ite courante partagerait plus de la moiti\'e de ses points valides avec la solution trouv\'ee.

   
Avantage : on s'arr\^ete plus t\^ot dans l'arbre de recherche, surtout quand il y a des plans avec des nombres
de mesures diff\'erentes.

Inconv\'enient : on n'a plus les solutions optimales en ensemble de points valides
   et on ne sait plus caract\'eriser les solutions rendues par le post-traitement (cf partie 8), cet ensemble
de solutions apr\`es post-traitement  d\'ependant alors de l'ordre dans lequel ces solutions ont \'et\'e trouv\'ees.

La seule chose que l'on sait est que les solutions non recherch\'ees sont proches des solutions
trouv\'ees, dans le sens qu'elles auraient partag\'e avant posttraitement la moiti\'e de leurs mesures
avec une solution trouv\'ee.

Cette nouvelle condition est actuellement optionnelle.

\section{Recherche d'outliers}

  cf mail du Chili  11/12

   La recherche est lanc\'ee si   possibles -Q  est inf\'erieur \`a un seuil.

   La recherche des ouliers s'effectue en prenant des triplets d'observations disjoints et en cherchant si la solution correspondante \`a un triplet donn\'e a une intersection  vide avec la bo\^ite courante, auquel cas on a trouv\'e qu'il y a un outlier dans ce triplet (mais on ne
sait pas de quelle observation il s'agit).
   On rejette la bo\^ite courante si on a trouv\'e plus d'outliers que la diff\'erence entre le nombre de mesures possibles et  Q. 
   Ce seuil est un  param\`etre de l'algorithme : dans les exemples, on a essay\'e de le mettre entre 0 et 20.
   La  diminution du nombre de boites assez sensible (20 \%), mais le temps de r\'esolution varie peu.
   L'algo ICTAI est obtenu en mettant ce param\`etre seuil = -1. (on n'effectue aucun test).

\section{Recherche de points valides}
\label{ptvalide}
\subsection{Limitation recherche point valide}
Par ailleurs, on a limit\'e le lancement de la recherche du point valide en accord avec la condition d'arr\^et:
 on ne lance cette recherche que si on a une possibilit\'e d'obtenir la condition d'arr\^et.
   

\subsection{Mini ransac}
  cf mail du Chili  11/12
    Au lieu de rechercher \`a valider les observations sur le point milieu de la bo\^ite courante, on essaye \`a partir de $k$ triplets
tir\'es al\'eatoirement. $k$, le nombre de triplets  est un param\`etre de l'algorithme.

\begin {itemize}
\item point milieu (k=0)  : calcul des mesures valides au point milieu de la bo\^ite courante  (algo ICTAI)
\item k :  chaque point pour la validation des observations est le milieu de l'intersection  de la boite courante avec la solution
	  de l'intersection de 3 mesures choisies al\'eatoirement (mini-ransac)
\end{itemize}     
Les meilleurs r\'esultats sont obtenus avec $k$ =1 , mais ils varient assez peu pour $k <= 3$.  Au del\`a, cette recherche est trop
ch\`ere.

\subsection{Limitation recherche point valide}
\label{limite-validation}
Par ailleurs, on a limit\'e le lancement de la recherche du point valide en accord avec la condition d'arr\^et:
 on ne lance cette recherche que si on a une possibilit\'e d'obtenir la condition d'arr\^et.
   

\section {Limitation des q-intersections}
\label{limite-qinter}
  (cf mail 24/12)
   Quand on choisit de faire des q-intersections dans la direction suppl\'ementaire, un
seuil est mis pour le lancement de la q-intersection sur les directions de la boite courante.
Ce seuil est determin\'e par :   possibles - Q.
Dans la version courante, le seuil est fix\'e \`a 10 pour les plans, c.\`a.d les q-intersections dans 
les directions de la bo\^ite courante ne sont faites qu'en bas de l'arbre de recherche quand   possibles - Q $<=$ seuil.



\section{Optimisation}
Noux avons ajout\'e un algorithme d'optimisation (maximisation) sur le nombre de mesures valides ({\it inliers}). C'est un algorithme
de type Branch and Bound avec plusieurs strat\'egies possibles : 
\begin{itemize}
\item une strat\'egie  en profondeur d'abord simple sans  choix du meilleur des 2 fils

\item une strat\'egie  en profondeur d'abord simple avec  choix du meilleur des 2 fils : 
 le crit\`ere de choix peut \^etre  le plus grand nombre d'observations valides ou la borne sup des observations de la boite 

\item une strat\'egie en meilleur d'abord.
\end{itemize}
La strat\'egie en meilleur d'abord cause assez rapidement des explosions m\'emoire dans les essais sur la matrice fondamentale, ainsi que le deuxi\`eme exemple de sc\`ene r\'eelle avec 3611 points (house44).


Un algo de type IbexOpt (contraction des fils et recherche point faisable  avant mise dans le tas) a \'et\'e implant\'e.

Pour les plans, cette strat\'egie en profondeur d'abord avec  choix du meilleur fils est la plus rapide, dans l'implantation
de type IbexOpt.

Autres strat\'egies ou autres crit\`eres \`a \'etudier ?  
\paragraph{Bornes}
Pour chaque bo\^ite, le majorant qmax du nombre d'inliers est calcul\'e par l'algorithme de q-intersection par projections comme le plus grand nombre de possibles sur un intervalle projet\'e contenant au moins q observations possibles. Ce majorant peut \^etre
plus petit que le nombre de possibles dans la bo\^ite courante (en particulier s'il y a plusieurs solutions dans la bo\^ite
courante).

Si ce majorant est plus petit que q, la branche est abandonn\'ee.

Le minorant est le nombre de valides.
La condition d'arr\^et de la branche courante est :  $qmax = valid$  ou pr\'ecision atteinte.

Le crit\`ere \`a optimiser \'etant un entier, quand on a trouv\'e une solution avec k inliers, on recherche par la
suite une solution avec au moins q=k+1 inliers.   (eps-sol de IbexOpt=1).

   Pas de post-traitement :  on obtient directement la meilleure solution et une valeur de la borne sup possible sur les petites 
boites ($qsb$).  (\'equivalent \`a uplo-of-epsboxes dans Ibexopt). Si cette borne est in\'erieure ou \'egale \`a la valeur de la meilleure solution, l'optimum est prouv\'e.

   Param\`etre de l'algorithme : pr\'ecision de bissection :  si elle est assez petite, on n'obtiendra pas de petites bo\^ites
avec de meilleures solutions potentielles et l'optimum obtenu sera prouv\'e, mais le temps de calcul risque d'\^etre grand.

   En cas d'arr\^et en limite de temps : on a une borne sup\'erieure sur les boites restantes.  (\'equivalent \`a uplo de Ibexopt), mais en profondeur d'abord, cette valeur est grande  : on ne fait rien pour la descendre.
Ce point n'est pas explicit\'e dans le pseudo-code.

   Cet algorithme d'optimisation marche bien sur les plans et les cercles : les optimums trouv\'es sont prouv\'es sur tous les exemples artificiels et sur les deux sc\`enes r\'eelles.

Tol\'erance sur l'objectif : au lieu d'une solution entière optimale , on peut rechercher l'optimum avec une tol\'erance entière epsobj donn\'ee (\'equivalent de $eps_{sol}$ de Ibexopt).

Limite des branchements : au delà d'une certaine profondeur, on ne poursuit qu'une branche, l'autre (si elle existe) est consid\'er\'ee comme une petite 
boîte et n'est pas mise dans le tas:  paramètre de l'algo : profondeur max pour 2 branches
choix de la branche : actuellement qmax .
peut-on faire mieux ?

\begin{algorithm}[htb]
\SetVline
{\small
\Algo { {\tt QinterOptim} ($box$, $observations$, $q$,
  $\epsilon_{sol}$, $\tau$) }
{
$solution$ $\gets \emptyset$ \;
$qsb=0$;
$node$ $\gets$ {\tt new} $Node$ \;
$node.box$ $\gets$ $box$ \;
$node.possibleCS$ $\gets$ $observations$ \;
$node.validCS$ $\gets \emptyset$ \;
$nodeStack$ $\gets$ $\{node\}$ \;

\While {$nodeStack$ $\ne \emptyset$} {
   $node$ $\gets$ {\tt pop} ($nodeStack$) \;
   $box$ $\gets$ $node.box$ \;
  {\tt bisect}  ($box$, $box_1$, $box_2$)  /* split the box */\;
  $node_1$ $\gets$ {\tt quasiCopy} ($node$, $box_1$) \; 
  $node_2$ $\gets$ {\tt quasiCopy} ($node$, $box_2$) \; 

  {\tt handleNode}($node_1$, $q$,
  $\epsilon_{sol}$, $\tau$ , $solution$, $qsb$) \;
  {\tt handleNode}($node_2$, $q$,
  $\epsilon_{sol}$, $\tau$ , $solution$, $qsb$) \;
  \eIf {$node_1$.isLeaf $=$ false {\tt and} $node_2$.isLeaf $=$ false}
{
  {\tt push} ($nodeStack$, worst($node_1$,$node_2$)) \; 
  {\tt push} ($nodeStack$, best($node_1$,$node_2$))
}
{ \eIf {$node_1$.isLeaf $=$ false}
  {{\tt push} ($nodeStack$,$node_1$)}
  {\If {$node_2$.isLeaf $=$ false}
  {{\tt push} ($nodeStack$,$node_2$)}
}
}
}
 % end while
\Return (solution, qsb)
}
}



\caption{Algorithme d'optimisation donnant une solution contenant le nombre maximum d'inliers : variante en profondeur d'abord avec choix du meilleur fils}
\label{algo:qinterestim}
\end{algorithm}



\begin{algorithm}[htb]
\SetVline
{\small
\Algo { {\tt handleNode} ($node$,  $q$,
  $\epsilon_{sol}$, $\tau$ , $solution$, $qsb$)}
{
  $node$.isLeaf $\gets$ false\;
$box$ $\gets$ $node.box$ \;
   
   {\tt contAndQinter} ($box$, $\tau$, $q$, $node.possibleCS$, $node.maxpossibles$ ) \;
   % contract the box and updates the compatible measurements for box.
                                    
   \eIf  {$box$ $\ne \emptyset$}
  % the current box is empty if qinter eliminates it or if the compatible measurements has a size
  % less than q 
    {
       {\tt validateBox} ($box$, $node$, $\tau$, $node.possibleCS$, $node.validCS$) \;

       \If  { $|node.validCS|$  $>=q$}
            { $solution$ $\gets$  $node$ \;
              $q$ $\gets$ $|node.validCS| +1$\;
              }\;
            \If {width($box$) $< \epsilon_{sol}$  {\bf or}  $|node.validCS|$ $=$ $node.maxpossibles$ }
                {
            \If {$node.maxpossibles$  $> |node.validCS|$}
            {$qsb$  $\gets$ max ($qsb$, $node.maxpossibles$)}
                
               $node$.isLeaf $\gets$ true
                }         
    }
     {$node$.isLeaf $\gets$ true}

}
}


\caption{Traitement d'un noeud : contraction, calcul des valides et test feuille}
\label{algo:qinterestim}
\end{algorithm}



\subsection {Questions ?}


Le branch and bound est classique. 
Les points originaux sont la d\'etermination du majorant de chaque bo\^ite par l'algo de q-intersection, m\'ethode g\'en\'erique ind\'ependante
de l'application.

et du point faisable par des ``mini-ransac''   (mais qui n'am\'eliore qu'\`a la marge la solution avec la validation sur le point milieu).


Les conclusions sur les am\'eliorations sont les m\^emes : les seules am\'eliorations vraiment gagnantes sont celles de la section \ref{limite-qinter} et celles de la section \ref{limite-validation}, qui suppriment des calculs inutiles.

Il faudrait de nouveaux exemples artificiels (plus r\'ealistes ?)   et r\'eels   pour \^etre convaincu de l'int\'er\^et de l'approche.
  
   Essais sur la matrice fondamentale avec 2 crit\`eres ( alg\'ebrique  et Sampson). On a des
   difficult\'es dues \`a la combinatoire (7 ddl): on n'arrive qu'\`a optimiser sur des petites boites autour d'une solution trouv\'ee par RANSAC : on am\'eliore localement cette solution.


Pour les probl\`emes avec contraintes, la recherche d'un point faisable se fait en construisant un polytope int\'erieur
(comme Ibexopt).  Question : que minimiser sur ce polytope dans un simplexe pour avoir le plus de points valides ??


\subsection {Travaux li\'es}

2 articles de ECCV 2014 qui r\'ealisation une maximisation de l'ensemble des inliers par branch and bound en vision.
\begin{itemize}
\item {rotation et distance focale  (dimension 4)} : 
Globally Optimal Inlier Set Maximization
With Unknown Rotation and Focal Length\\
Jean-Charles Bazin, Yongduek Seo, Richard Hartley, Marc Pollefeys   4DDL
\item {Matrice essentielle (dimension 5) } :
Optimal Essential Matrix Estimation via
Inlier-Set Maximization  \\
  Yang , Li, Jia ECCV 2014  

\end{itemize}




\section {Nouveaux exemples artificiels de plans }

\begin{itemize}

   \item points compl\`etement al\'eatoires (que du bruit)
     permet de v\'erifier que l'algorithme a, \`a Q fix\'e, un comportement en $m^3$ , et que la complexit\'e
diminue avec Q.

  \item  nouveaux exemples avec 25 plans
\begin{itemize}
      \item     sans coordonn\'ees enti\`eres : (les exemples ICTAI avaient malencontreusement les coordonn\'ees al\'eatoires enti\`eres)   plus de petites boites avec strat\'egie round robin

      \item sans plans parall\`eles ni coordonn\'ees enti\`eres : la recherche de points valides sur ces exemples
permet de s'affranchir de la pr\'ecision.  il n'y a plus de petites boites (avec strat\'egie round robin ou sp\'ecifique rr2)
\end{itemize}
   
\end{itemize}

\section {Triplets de mesures}

    R\'esolution avec tous les triplets :  complexit\'e $m^3$ en temps et en m\'emoire 
due \`a la phase de construction des boites pour tous les triplets.  
    Cette phase est constante \`a $m$ donn\'e (ind\'ependante de Q) et rend cet algorithme peu efficace.
    Contrairement \`a l'algorithme d'ICTAI, il ne peut tirer partie du seuil Q.

    


    
\section  {Propri\'et\'es des solutions}

Suivant la condition d'arr\^et utilis\'ee, on a les propri\'et\'es suivantes sur les solutions obtenues.


\subsection {Condition d'arr\^et : pr\'ecision ou possibles==valides}
       solutions : solutions trouv\'ees maximales dans la boite courante (valides $\geq Q$ et valides=possibles) 
       $\cup$  solutions trouv\'ees  non prouv\'ees maximales dans la boite courante (petites boites avec possibles $\geq Q$ et valides $\geq Q$  possibles)
       $cup$  solutions potentielles (petites bo\^ites avec  possibles $\geq Q$  et valides $< Q$)


       Le posttraitement ne porte que les solutions trouv\'ees. Parmi celles-ci:
\begin{itemize}
\item{ On garde les solutions maximales}.

\item si 2 solutions partagent la moiti\'e de leurs points , on  garde la meilleure  (ordre total)
             crit\`ere principal ; nombre d'inliers 
             crit\`ere secondaire somme des valeurs absolues de la fonction $f$ (compatibilit\'e des observations).
       
\end{itemize}
Propri\'et\'e des solutions gard\'ees : elles sont maximales parmi les solutions trouv\'ees.

Propri\'ete garantie des solutions retir\'ees : elles  partagent la moiti\'e des points avec une solution gard\'ee.
En triant les solutions selon le crit\`ere ci-dessus, l'algo de retrait garantit cette propri\'et\'e.


 \subsection {Condition d'arr\^et :  pr\'ecision ou   possibles-valides < q/2}

       solutions = solutions trouv\'ees  $\cup$ solutions potentielles (petites boites) \\
                                       $\cup$ solutions potentielles partageant 1/2 des solutions trouv\'ees.

       Les solutions trouv\'ees ne comprennent plus les solutions optimales m\^eme quand il n'y a pas
de petites boites.

       Mais on est s\^ur que pour chaque solution trouv\'ee, les solutions potentielles dans les sous-boites de la
       boite de la solution trouv\'ee partagent  au moins la moiti\'e des points de cette solution.


      m\^eme post-traitement sur les solutions trouv\'ees.


\section{R\'esultats exp\'erimentaux}


\subsection{Plans et cercles}
On a compar\'e les r\'esultats obtenus avec les diverses modifications avec ceux d'ICTAI.

\paragraph{Toutes solutions}
Les tests ont \'et\'e les suivants :

\begin{itemize}
\item{ibex}  : algo ICTAI dans nouvelle version Ibex.  Cette nouvelle version est utilis\'ee dans la suite.
\item{m3.1} : limitation de la recherche de points valides; 
cette limitation est ensuite appliqu\'ee \`a tous les autres tests.
\item{m4} : limitation dans les appels \`a la q-intersection
\item{m14} : limitation dans les appels \`a la q-intersection et nouvelle condition d'arr\^et.

\item{m2(10)34}  : recherche outliers (seuil 10) , limitation dans les appels \`a la q-intersection, 1 appel \`a mini-ransac
\item{m2(0)34}  : recherche outliers (seuil 0) , limitation dans les appels \`a la q-intersection, 1 appel \`a mini-ransac
\item{m2(10)34b}:recherche outliers (seuil 10) , qintersection aff uniquement, 1 appel \`a mini-ransac
\end{itemize}








\begin{table*}[htb]
\caption{Recherche de solutions : Temps cpu}
\begin{center}
%{\scriptsize
\begin{tabular}{|l||r|r|r|r|r|r|r|r|r|r||r|r|}
\hline
Test case & $P_1$ & $P_2$ & $P_3$ & $P_4$ & $P_5$ & $P_6$ & $P_7$ & $P_8$ & $P_9$ & $H_{40}$ & $C_1$ & $C_2$\\
\hline
ICTAI &1.1& 2.4   & 3.8  &18.3 & 36.4 &107.5&   53.7  &  92.5 &   221.4 & 24.5 &2.3 &6.0\\
ibex & 1.0 & 2.5 & 4.1 & 19.6 & 38.6 & 113.6 & 59.1 &103.2 & 245.2 & 24.5 & 2.2 & 5.9 \\
m3.1 & 1.0 & 2.3 & 3.7 & 17.5 & 34.5 & 102.0 & 53.4 & 92.4 & 221.3 & 23.1 & 2.1 & 5.7\\
m4  & 0.8  & 1.7 & 2.8 & 13.5 & 26.5 & 79.1 & 37.9 & 65.6 & 156.1 & 20.8 & 2.0& 5.4\\
m14&0.7 & 1.7 & 2.7 & 13.5 & 26.5 & 79.4 & 38.0 & 65.3 & 157.2 & 17.9 & 1.7 & 5.2 \\
m2(10)34 & 0.8 & 1.7 &  2.7 & 13.4 & 26.2 & 78.0 & 38.0 & 65.0 & 155.0 & 21.3 & 2.0 & 5.4 \\
m2(0)34 &    0.8 & 1.7 &  2.7 & 13.5 & 26.7 & 79.3 & 38.1 & 65.5 & 157.2 & 20.9 & 1.6 & 5.0  \\
m2(10)34b & 0.8&  1.7&   2.7  &  13.3 & 26.0 & 77.3 & 38.0 & 65.9 & 156.7 & 21.7 & 2.1  & 5.7 \\
\hline
\end{tabular}
%}
\end{center}
\label{planeresults}
\end{table*}

\begin{table*}[htb]
\caption{Recherche de solutions : Nombre de noeuds}
\begin{center}
{\tiny
\begin{tabular}{|l||r|r|r|r|r|r|r|r|r|r||r|r|}
\hline
Test case & $P_1$ & $P_2$ & $P_3$ & $P_4$ & $P_5$ & $P_6$ & $P_7$ & $P_8$ & $P_9$ & $H_{40}$ & $C_1$ & $C_2$\\
\hline

ICTAI & 4,786  &11,626& 20,334  &122,590  &285,898 & 1,226,748 &   61,786  &  111,558  & 293,834 &646,694&30,344 &  40,600\\
ibex & 4,786 & 11,626 &20,334 & 122,558 &285,838 & 1,226,652 & 61,766 & 111,528 & & 646,694 &  26,838 &37,096 \\
m3.1 & 4,786  &11,626& 20,334  &122,590 &285,890 &  1,226,728 & 61,786 &111,558&293,834 &646,694 & 26,838 & 37,096\\
m4 & 5,262& 12,038 & 21,034 & 134,280 & 313,314 &1,353,368 & 63,918 & 115,822 &307,124 & 659,772 &27,264 & 39,366 \\
m14&4,136 & 11,332 &20,376 & 133,968 & 312,738 & 1,352,362 & 63,652 & 115,494 & 306,554& 544,412 & 22,860 & 34,962  \\
m2(0)34 & 5,250 & 11,794 &20,606  &124,510   & 283,040  &1,136,310   &63,062 & 113,202 &296,856 &636,978  & 16518  & 28306 \\
m2(10)34 & 5,164 & 11,136 & 19,502 & 109,872 & 252,276 & 1,023,700 &60,780 & 106,776 &273, 244 &630,556 & 16,304 & 27,530 \\
m2(10)34b &5,372 & 11,418 & 19,906 &111,110 &254,060 & 1,027,526 & 61,680 &107,832 &  274,404 & 693,372 & 26,838 &  37,096 \\
\hline
\end{tabular}
}
\end{center}
\label{noderesults}
\end{table*}

\paragraph{Optimization}

Les exp\'erimentations suivantes ont \'et\'e r\'ealis\'ees :

\begin{itemize}
\item{ibex} : algo optimisation dans nouvelle version ibex , sans am\'eliorations post-ictai
\item{m3.1} : limitation calcul points valides
\item {m3.1m4} : limitation calcul points valides + limitation q-intersection
\item {m3.1m4m2(10)}: recherche outliers (seuil 10) , limitation dans les appels \`a la q-intersection et points validess , 1 appel \`a mini-ransac
\item {ibexopt bs m3.1} profondeur meilleur fils et toutes am\'eliorations
\item {ibexopt bs no3.1} profondeur meilleur fils et toutes am\'eliorations sauf limitation points valides
\item{bfs no3.1} : strat\'egie meilleur d'abord  et toutes am\'eliorations sauf limitation points valides
\item{bfs m3.1} : strat\'egie meilleur d'abord et  toutes am\'eliorations 
\end{itemize}

\begin{table*}[htb]
\caption{Optimisation : Temps cpu}
\begin{center}
{\scriptsize
\begin{tabular}{|l||r|r|r|r|r|r|r|r|r|r||r|r|r|}
\hline
Test case & $P_1$ & $P_2$ & $P_3$ & $P_4$ & $P_5$ & $P_6$ & $P_7$ & $P_8$ & $P_9$ & $H_{40}$ & $H_{44}$ &  $C_1$ & $C_2$\\
\hline
ibex &0.3& 0.9   & 1.1 &11.0 & 16.9 &35.8& 43.2 & 65.8 &  132.2 & 1.7 & 454.3& 0.9& 4.5 \\
m3.1  & 0.3 & 0.8& 1.0 & 9.9 & 15.2 & 32.4 & 39.4 & 60.3 & 121.4&  1.5 & 412.6 & 0.8 & 4.4 \\
m3.1m4&0.3 &0.7 & 0.9 &9.0 & 14.1 &30.2 &32.8 &49.8 &100.9 &1.3 &322.4 &0.8 & 4.4 \\
m3.1 m4 m2(10)&0.2 &0.7 &0.8 &7.7 & 12.1 &26.1 &28.5 &43.3 &88.0 &1.1 & 252.4 & 0.8 & 4.4 \\
ibexopt bs no3.1&0.2  &0.5 & 0.7 & 6.4 &9.1 &18.1 &26.6 &39.0 &75.5 &1.2 &142.5 &0.7& 4.5 \\
ibexopt bs m3.1& 0.2 & 0.5 & 0.7 & 6.3 & 8.8 & 17.8 & 26.1  &38.6  & 73.2 &1.2  &140.6 &0.7 &4.1 \\
bfs no3.1 & 0.2  & 0.5 & 0.5 & 7.9 & 12.0 & 28.5 & 29.7 & 45.3 & 91.8 & 1.0 & 193.9& 0.7& 4.6\\
bfs m3.1& 0.2  & 0.4 & 0.4 & 7.8 & 12.0 & 28.5 & 29.5 & 45.9 & 91.3 & 1.0 & MO & 0.7& 4.2\\

\hline
\end{tabular}
}
\end{center}
\label{optimresults}
\end{table*}

Pour les maisons H40 et H44, en optimisation, une pr\'ecision à 0.0001 permet d'obtenir l'optimum sans avoir de petites boîtes.  optimum à 38 inliers pour H40 et 171 inliers pour H44 avec une tol\'erance de 0.001.

probl\`eme r\'esolu en 1.2s pour H40 et 166s pour H44  avec ibexopt profondeur ( ibexopt bs no3.1)

\subsection{Matrice fondamentale}

Crit\`ere alg\'ebrique. Essais avec une bo\^ite initiale autour de la solution  144    (solution Ransac 146).
Mod\'elisation \`a 8 variables (v(3,3)=1), tol\'erance 0.001, points normalis\'es.

Essais suivants r\'ealis\'es avant la mise en place du polytope int\'erieur pour la contrainte de d\'eterminant
qui semble allonger un peu le temps de calcul (25\% sur un exemple) 


Par ailleurs, une solution avec Ransac \`a 153 a \'et\'e trouv\'ee.
\begin {itemize}
\item {petites boites}\\
        taille  0.000003  144  prouv\'e en 16s prec 0.0000005       BF  0.46s \\
        taille  0.000004  145   prouv\'e en 87s  prec 0.0000005        BF 1.4s    NBF 0.44s 1246 \\
        taille 0.000012   146   prouv\'e en 164s.  prec 0.0000005     BF  436s. 60M   NBF 166s. \\
        taille 0.000013   146   prouv\'e en 536s   prec 0.0000002\\
        taille 0.000015   147   prouv\'e en 2149s  prec 0.0000005 \\
        taille 0.000018   147   prouv\'e en 84s (91s)   prec 0.0000005    NBF 45s  117148 noeuds \\
        taille 0.00002    147   prouv\'e en 149s   prec 0.0000005         NBF 102s 265984 noeuds  \\
        taille 0.000025   147   prouv\'e en 691s   prec 0.0000005         NBF 711s 1700732 noeuds \\
        taille 0.00003    146   non prouv\'e en 10000s.\  NBF 147  prouv\'e en 2343s.  5870534 noeuds \\
        taille 0.00004    146   non prouv\'e en 10000s.

\item {taille  0.001   }
           prec 0.001  118 valides      200 possibles    27s  \\
           prec 0.0008 126 valides      197 possibles    53s  \\
           prec 0.0007 131 valides      196 possibles    57s  \\
           prec 0.0006 135 valides      195 possibles    62s  \\
           prec 0.0005 145 valides      187 possibles   1963s  \\
           prec 0.00049 147 valides     185 possibles   2958s  \\
           prec 0.00048 147 valides     185 possibles   2983s  \\
           prec 0.00045 147 valides     185 possibles   3003s  \\
           prec 0.0004  147 valides     185 possibles   3475s  \\
           prec 0.0003  148 valides     185 possibles   4211s  \\
           prec 0.0002  148 valides   non prouv\'e       10000s      NBF 152 puis MO \\


\item {taille 0.002}
          prec 0.001   142 valides      210 possibles  2741s.  \\
          prec 0.0006  140 valides      198 possibles   4597s. \\

\item{taille 0.01}
         prec 0.01     27 valides      356 possibles   17s  \\
         prec 0.009    40 valides      334 possibles  185s  \\
         prec 0.008    40 valides      334 possibles  195s. \\
         prec 0.007    46 valides      329 possibles  250s. \\
         prec 0.006    46 valides      329 possibles  299s.  \\
         prec 0.005    81 valides      314 possibles 5639s. \\
         prec 0.004    78 valides      $>$ 10000s    \\
         prec 0.003    79 valides      $>$ 10000s   NBF 68 en 2mn , puis 81 en 7m , puis 82   \\
         prec 0.002   108 valides      $>$ 10000s\\

\item{taille 0.02}
         prec 0.01     43 valides       364 possibles 7860s.   

         prec 0.001    NBF 83 valides   MO \\

\item{taille 0.05}
         prec 0.01      22 valides       NBF 77 valides  MO\\

\end{itemize}

\subsection{Matrice essentielle}

Mod\'elisation 9 variables avec contraintes de matrice essentielle (2 ddl) , de norme (1 ddl) , de d\'eterminant (1 ddl)

\subsubsection {Optimisation sous contraintes}

R\'ealisation d'un optimiseur en profondeur d'abord guid\'e par un oracle. (solution de RANSAC).


\begin{itemize}
\item {Propagation des contraintes de matrice essentielle} :  relaxation lin\'eaire (polytope hull) avec 3 contraintes de matrice essentielle sur 9, la norme et le d\'eterminant et lin\'earisation avec arithm\'etique affine. 

\item{Recherche de point valide} : polytope int\'erieur  coin al\'eatoire pour chaque contrainte et  minimisation al\'eatoire d'une direction. Au maximum 10 essais par noeud  (tant que la zone faisable est non vide).

\end{itemize}


\subsubsection{Q-inter}

2 crit\`eres  :  alg\'ebrique et Sampson

\paragraph{Crit\`ere alg\'ebrique} 

La  r\'esolution termine sur l'exemple test2 de usac,  avec une tol\'erance de 0.01, on obtient les r\'esultats suivants :
on utilise la projection  suppl\'ementaire.
\begin{itemize}
\item  pr\'ecision 0.005  [516 520] en  21886s.

\item pr\'ecision 0.01  [516 524] en  14026s.

\item pr\'ecision 0.02  [514 528] en 11390s.

\end{itemize}

avec hc4 sur les contraintes , y compris les contraintes redondantes, la bissection smearsumrel,  on 
obtient les r\'esultats suivants 


\begin{itemize}
\item pr\'ecision  0.001  [516 517] en 46500s

\item pr\'ecision  0.0025  [516 518] en 26352s

\item  pr\'ecision 0.005  [516 520] en  17580s.

\item pr\'ecision 0.01  [516 525] en  10022s.
\item pr\'ecision 0.01  1 branche prof 48 [515 530] en  9231s.
\item pr\'ecision 0.01  1 branche prof 48 [515 537] en  6968s.  3930566 branches
\item pr\'ecision 0.01  1 branche prof 40 [514 544] en  4630s.  2643136 branches

\item pr\'ecision 0.01  1 branche prof 40 [513 521] en  5689s.  2982596 branches  critère valides-pt milieu

\item pr\'ecision 0.02  [514 528] en 11390s.

borne sup fausse
\item pr\'ecision 0.0005  [515 516] en 5158s.  2626202 branches  prof 40 critère valides-pt milieu
\item pr\'ecision 0.001  [515 516] en 5560s.  2619290 branches  prof 40 critère valides-pt milieu
\item pr\'ecision 0.002  [514 515] en 5504s.  2790810 branches  prof 40 critère valides-pt milieu
\item pr\'ecision 0.0002  [514 515] en 5473s.  2795200 branches  prof 40 critère valides-pt milieu


\end{itemize}

avec une precision de 5 sur l'objectif
\begin{itemize}
\item  pr\'ecision 0.005  [515 521] en  5768s.


\item pr\'ecision 0.01  [515 524] en  4567s.
\item pr\'ecision 0.01 1 branche prof 40 [512 544] en 3633s.
\item pr\'ecision 0.01 crit\`ere milieu 1 branche prof 40 [512 521] en 4412s et  2324468 branches


\item pr\'ecision 0.02  [512 528] en 6786s.

\end{itemize}




estim_essentialmatnorm essential_test2.txt 10 0.001 2 0.015 1.e-4 0 0 100000 2 
avec une tol\'erance 0.001, c'est plus long.


\paragraph{Crit\`ere de Sampson}

Pour Sampson, mise en place 
sur exemple test2 de usac (582 correspondances),  on trouve une solution autour de 424 - 431  inliers  avec une tol\'erance de 1.

plus ou moins rapidement selon l'oracle de d\'epart  (solution RANSAC)

avec oracle 279 inliers, on trouve rapidement (4mn) une solution \`a 330-340 inliers , puis la solution \`a 420 - 430 inliers en 50000s
avec pr\'ecision 0.01

Mise en place d'un seuil sur la q-inter et fwdbwd
avant seuil ($>$ 0.1)  fwd uniquement , apr\`es seuil  fwdbwd + q-inter
m\^emes r\'esultats : solution 424 en 12h.


Mise en place de la projection affine : cela ne semble pas rentable (co\^ut trop important  de l'arithm\'etique affine)  : 316 en 20mn,
pas mieux en 12h. seuil \`a envisager ??








avec prec0.02  solution 388 en 12h.




sur petit problème (36 mesures), la projection affine semble rentable :  solution à 29 trouvée avec epsobj5 avec domaine total en 3600s.

(pb fermé en [28 36] avec epsobj6  sans proj affine en 14689s.)
pb fermé en [30 34] avec epsobj3  prec 0.005 , limite bissection profondeur 60, avec proj affine en 48595s.



\end{document}
